\documentclass{article}
\usepackage[utf8]{inputenc}
\usepackage{amsmath} % Para entornos matemáticos, aunque no estrictamente necesario para esta tabla.
\usepackage{graphicx} % Para incluir gráficos si fuera necesario.
\usepackage{booktabs} % Para tablas más profesionales (toprule, midrule, bottomrule).
\usepackage{hyperref} % Para enlaces clicables en las referencias URL.
\usepackage{geometry} % Para configurar márgenes.
 \geometry{a4paper, margin=1in} % Márgenes de 1 pulgada

\title{Estación de Trabajo para Simulación Científica: Aplicación a la Climatología}
\author{Angel Besteiro}
\date{\today} % Esto pondrá la fecha actual de compilación

\begin{document}

\maketitle % Esto genera el título y el autor/fecha.

\section*{Introducción al Proyecto}

El desarrollo de avances en distintas áreas trae consigo números y planteamientos que probar. Normalmente, estos se resuelven y analizan con pruebas de campo o el método de prueba y error. El problema es que, en ciertas situaciones y con determinados problemas, realizar muchas pruebas en el mundo real puede ser costoso, poco práctico y profundamente ineficiente.

He querido orientar aún más el tema a la \textbf{climatología}, ya que no veo un mejor ejemplo de una rama que necesite y se complemente tan bien con simulaciones a escala por computadora. El hecho de revisar los cielos de determinados lugares sin que siquiera hayan pasado horas es crucial. Los modelos climatológicos actuales cumplen con lo siguiente:
\begin{itemize}
    \item \textbf{Modelos matemáticos:} Utilizan ecuaciones complejas para replicar la física atmosférica.
    \item \textbf{Equipo Adecuado:} Ejecutan cálculos intensivos para procesar grandes cantidades de datos.
    \item \textbf{Datos de entrada:} Información meteorológica obtenida de satélites, radares y estaciones meteorológicas.
    \item \textbf{Visualización:} Herramientas que permiten analizar y representar gráficamente los resultados.
\end{itemize}

---

\section*{Configuración de la Estación de Trabajo}

A continuación, se detalla la selección de componentes para la estación de trabajo orientada a la simulación científica, específicamente diseñada para tareas de climatología, bajo un presupuesto de \$2500, buscando optimizar al máximo la inversión.

\begin{table}[h!]
    \centering
    \caption{Componentes de la Estación de Trabajo para Simulación Científica}
    \label{tab:pc_components}
    \begin{tabular}{llc}
        \toprule
        \textbf{Componente} & \textbf{Precio (USD)} & \textbf{Referencia} \\
        \midrule
        GPU (Gigabyte GeForce RTX 4070 Ti SUPER 16GB) & \$1035 & \href{https://www.sigmatiendas.com}{sigmatiendas.com} \\
        CPU (AMD Ryzen 9 9900X) & \$419 & \href{https://www.antonline.com}{antonline.com} \\
        RAM (128GB DDR5) & \$250 & \href{https://www.amazon.com}{amazon.com} \\
        Disipador (ARCTIC Liquid Freezer III 360) & \$118 & \href{https://www.amazon.com}{amazon.com} \\
        Placa Base (ASUS TUF Gaming B650-PLUS WIFI) & \$250 & \href{https://www.amazon.com}{amazon.com} \\
        SSD Principal (Western Digital SN850X 2TB NVMe) & \$160 & \href{https://www.westerndigital.com}{westerndigital.com} \\
        SSD Secundario (Crucial P5 Plus 1TB NVMe) & \$90 & \href{https://www.pccomponentes.com}{pccomponentes.com} \\
        Fuente de Poder (MSI MPG A850G 850W) & \$175 & \href{https://www.amazon.com}{amazon.com} \\
        \midrule
        \textbf{Total Estimado} & \textbf{\$2497} & \\
        \bottomrule
    \end{tabular}
\end{table}

\end{document}
